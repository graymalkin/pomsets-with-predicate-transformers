\documentclass[a4paper]{article}
\usepackage[utf8]{inputenc}
\usepackage[english]{babel}
\usepackage[margin=0.5in]{geometry}

\usepackage{amsmath,amssymb,stmaryrd,amsthm, stackengine}
\usepackage[hidelinks]{hyperref}
\usepackage{todonotes}
\usepackage{caption}
\usepackage{subcaption}
\usepackage{float}
\newcommand{\ie}{\emph{i.e.}}
\newcommand{\eqdef}{\triangleq}

% Notation for the Semantic structures
\newcommand{\haslabel}[2]{\{ #1 : #2 \}}
\newcommand{\squiggle}{\rightsquigarrow}
\newcommand{\rul}[3]{\frac{#1}{#2}\;#3}
\newcommand{\setcomp}[2]{\{#1 \mid\; #2 \}}
\newcommand{\downclosure}[1]{\lceil #1 \rceil}
\newcommand{\roots}[1]{\lfloor #1 \rfloor}
\newcommand{\closure}[1]{#1^{*}}
\newcommand{\closureplus}[1]{#1^{+}}
\newcommand{\acyclically}{\lesssim}
\newcommand{\den}[1]{\llbracket #1 \rrbracket}
\newcommand{\sem}[4]{\den{#1}_{#2\;#3\;#4}}
\newcommand{\emptysem}[1]{\den{#1}_{\emptyset\;\emptyset\;\emptyset}}
\newcommand{\bden}[1]{\llparenthesis #1 \rrparenthesis}
\newcommand{\sbind}{\bullet}
\newcommand{\sparallel}{\parallel}
\newcommand{\loc}{\textsc{l}}
\newcommand{\glo}{\textsc{g}}
\newcommand{\strue}{\textsc{T}}
\newcommand{\sfalse}{\textsc{F}}
\newcommand{\readact}{\textsc{R}}
\newcommand{\writeact}{\textsc{W}}
\newcommand{\relact}{\textsc{rel}}
\newcommand{\acqact}{\textsc{acq}}
\newcommand{\fencecmd}[1]{{\textsc F(} #1 {\textsc )}}
\newcommand{\readcmd}[2]{\readact\;#1\;#2}
\newcommand{\writecmd}[2]{\writeact\;#1\;#2}
\newcommand{\wocmd}[4]{\writeact^{\text{#1}}_{#2}\;#3\;#4}
\newcommand{\rocmd}[4]{\readact^{\text{#1}}_{#2}\;#3\;#4}
\newcommand{\printcmd}[2]{{\tt print}\;#2}
\newcommand{\Var}{\mathcal{X}}
\newcommand{\Reg}{\mathcal{R}}
\newcommand{\Val}{\mathcal{V}}
\newcommand{\ES}{\mathcal{ES}}
\newcommand{\context}{\textsc{Contx}}
\newcommand{\lfp}{\textsf{lfp}}
\newcommand{\cond}{\textsf{if}}
\newcommand{\true}{\textsf{true}}
\newcommand{\false}{\textsf{false}}
\newcommand{\justifies}{\vdash}
\newcommand{\notjustifies}{\nvdash}
\newcommand{\justifiesR}{\vdash_{\readact}}
\newcommand{\justifiesW}{\vdash_{\writeact}}
\newcommand{\justifiesP}{\vdash_{\times}}
\newcommand{\Con}[1]{\textsf{Con}(#1)}
\newcommand{\Conf}{\textsf{Conf}}
\newcommand{\preorder}{\sqsubseteq}
\newcommand{\porder}{\sqsubseteq}
\newcommand{\partfin}{\mathcal{P}_\text{fin}}
\newcommand{\conflict}{\ensuremath{\textsc{\#}}}
\newcommand{\lub}{\bigsqcup\limits_{n \in \omega}}
\newcommand{\configurations}[1]{\mathcal{F}(#1)}
\newcommand{\CPO}{\textsf{c.p.o.}}
\newcommand{\with}{\text{ with }}
\newcommand{\tup}[1]{\langle #1 \rangle} 
\newcommand{\tord}{\le}
\newcommand{\ctxext}[1]{\uparrow (#1)}
\newcommand{\wctxext}[1]{\uparrow_{\tord}(#1)}
\newcommand{\ctxextbare}{\uparrow}
\newcommand{\wctxextbare}{\uparrow_\texttt{writes}}
\newcommand{\cons}{\texttt{\cdot}}
\newcommand{\nil}{\texttt{nil}}
\newcommand{\ctx}{\mathsf{Ctx}}
% % \let\temptt\texttt
% \renewcommand\texttt[1]{\mbox{\temptt{#1}}}
% Notation for the Syntax
% Notation for the Syntax
\newcommand{\textttt}[1]{\texttt{\textup{#1}}}
\newcommand{\init}{\textttt{Init}}
\newcommand{\done}{\textttt{skip}}
\newcommand{\assign}[3][rlx]{#2\;\textttt{:=}\textsubscript{\textsc{#1}}\;#3}
\newcommand{\reg}[1]{\textttt{r}_{#1}}
\newcommand{\var}[1]{\textttt{#1}}
\newcommand{\equals}{\textttt{==}}
\newcommand{\smallerthan}{\textttt{<}}
\newcommand{\greaterthan}{\textttt{>}}
\newcommand{\greaterethan}{\textttt{>=}}
\newcommand{\smallerethan}{\textttt{<=}}
\newcommand{\mult}{\textttt{*}}
\newcommand{\add}{\textttt{+}}
\newcommand{\minus}{\textttt{-}}
\newcommand{\divis}{\textttt{/}}
\newcommand{\ifthen}[2]{\textttt{if(}#1\textttt{)\{}#2 \textttt{\}}}
\renewcommand{\ifthenelse}[3]{\textttt{if(}#1\textttt{)\{}#2\textttt{\}\{}#3\textttt{\}}}
\newcommand{\spaciousifthenelse}[3]{\textttt{if(}#1\textttt{)\{}#2\textttt{\}\{}#3\textttt{\}}}
\newcommand{\alignedif}[3]{\begin{aligned}&\textttt{if(}#1\textttt{)\{}
    \\ &\quad \;#2\\ & \textttt{\} else \{}\\ & \quad #3 \\
    &\textttt\}\end{aligned}}
\newcommand{\alignedifthenelse}[3]{\alignedif{#1}{#2}{#3}}
\newcommand{\alignedifthen}[2]{\begin{aligned} & \textttt{if(}#1\textttt{)\{}
    \\ & \quad \;#2\\ & \textttt{\}} \end{aligned}}
\newcommand{\fence}[1]{{\texttt F\texttt{(}}#1{\texttt )}}
\newcommand{\rel}{\textttt{release}}
\newcommand{\acq}{\textttt{acquire}}
\newcommand{\logand}{\&}
\newcommand{\sequencing}{\textttt{;}\;}
\newcommand{\seq}[2]{#1\sequencing #2}
\newcommand{\para}[2]{#1 \textttt{\parallel} #2}
\newcommand{\while}[2]{\textttt{while(}#1\textttt{)\{}#2\textttt{\}}}

\newcommand{\rf}{\textit{rf}}
\newcommand{\rmw}{\textit{rmw}}
\newcommand{\dep}{\textit{dep}}
\newcommand{\lck}{\textit{lck}}
\newcommand{\co}{\textit{co}}
\newcommand{\ppo}{\textit{ppo}}


% TiKZ Styles & Setup
\usepackage{tikz}
\usetikzlibrary{decorations.pathmorphing}
\usetikzlibrary{graphdrawing,graphs}
\usegdlibrary{trees}
\tikzstyle{basic} = [rectangle, rounded corners=3pt, thin,draw=black,
fill=blue!8, sibling distance=20mm, minimum width=11mm, minimum
height=5.5mm,inner sep=0pt]
\tikzstyle{indep} = [rectangle, rounded corners=3pt, very thick,
dashed, draw=green!50!black,
fill=blue!8, sibling distance=20mm, minimum width=11mm, minimum height=5.5mm,inner sep=0pt]
\tikzstyle{faded} = [rectangle, rounded corners=3pt, thin,draw=black!87, text=black!87, fill=blue!7, sibling distance=20mm]
\tikzstyle{bold} = [rectangle, rounded corners=3pt, thin, draw=black, fill=orange!45, sibling distance=20mm]
\tikzstyle{invisible} = [draw=none]
\tikzstyle{po} = [->, draw=black]
\tikzstyle{pob} = [->, draw=black,line width=1.300]
\tikzstyle{conf} = [-, draw=black!30!red, line width=0.9, decorate,
decoration={zigzag, segment length=7, amplitude=2.6, pre=lineto, pre length=0pt, post=lineto, post length=0pt}]
\tikzstyle{just} = [->, draw=green!50!black,line width=1.000, dashed]
\tikzstyle{ppo} = [->, draw=blue!50!black,line width=1.000, dashed]
\tikzstyle{justf} = [->, draw=green!15!orange!90!,line width=1.000, dashed]
\tikzstyle{rf} = [->, draw=red!80!black,line width=1.000]
\tikzstyle{rff} = [->, draw=red!15!black,line width=1.000]
\tikzstyle{lab} = [xshift=0.5cm,yshift=-0.1cm,color=green!50!black,font=\scriptsize]
\tikzstyle{labright} = [xshift=0.5cm,yshift=-0.1cm,color=green!50!black,font=\scriptsize]
\tikzstyle{lableft} = [xshift=-0.5cm, yshift=-0.1cm,color=green!50!black,font=\scriptsize]

\tikzstyle{lck} = [->, draw=blue!80!white,line width=0.5]
\tikzstyle{dep} = [->, draw=purple!80!black,line width=1.000]
\tikzstyle{co} = [->, draw=blue!20!black,line width=1.000]
\tikzstyle{rmw} = [->, draw=red!20!black,line width=1.000]
\date{}
\author{}
